\documentclass[12pt,a4paper]{article}
\usepackage[utf8]{inputenc}
\usepackage[brazil]{babel}
\usepackage[T1]{fontenc}

\usepackage{times}

\title{Exercícios sobre Gerência de Memória Virtual e Paginação.}
\author{Universidade Estadual do Centro-Oeste (UNICENTRO)\\
        Departamento de Ciência da Computação (DECOMP)\\
        Sistemas Operacionais II \\
        Professor Andres Jessé \\
	Paulo R. Urio }

\newcommand{\pergunta}[1]{\noindent \textbf{#1}}

\begin{document}

\maketitle

\pergunta{1. Explique a diferença entre endereços físicos e virtuais.}

Endereço físico, também chamado de endereço real e binário, é um número
que representa uma posição específica no dispositivo de memória.

O endereço virtual, também chamado de endereço lógico,  é um número que 
não reflete no valor real da memória e sim o valor usado para o mecanismo 
de mapeamento.

\pergunta {2. Explique a diferença entre fragmentação interna e externa. Em 
quais situações elas podem ocorrer?}

A fragmentação interna ocorre quando os programas não preenchem as partições 
onde são carregados, ocorre com técnicas de alocação absoluta, relocável
e contígua. A fragmentação externa ocorre em técnicas de alocação dinâmica
quando programas são terminados sem serem completamente liberados da memó-
ria, deixando cada vez espaços menores na memória, assim novos programas não
podem ser executados.

\pergunta{3. Explique o funcionamento do algoritmo ``worst-fit''. Qual a 
vantagem de utilizá-lo visto que ele sempre escolhe o ``pior'' caso?}

No worst-fit, o gerenciador de memória coloca o processo no maior
bloco de memória não alocado. A ideia nesta estratégia é que após a alocação
deste processo, irá sobrar a maior quantidade memória após o processo, 
aumentando a possibilidade de, comparado ao best-fit, outro processo poder 
usar o espaço restante.  O worst-fit tende a causar menos fragmentações.

\pergunta{4. Dadas particões de memória de 100K, 500K, 200K, 300K e 600K (nesta 
ordem), como cada um dos algoritmos ``first-fit'', ``best-fit'' e ``worst-fit''
colocaria os processos de 212K, 417K, 112K e 426K (nesta ordem)? Qual algoritmo 
faz uso mais eficiente da memória?}

Sendo: P1 = 212K, P2 = 417K, P3 = 112K e P4 = 426K.

\begin{table}[!ht]
\begin{center}
	\begin{tabular}{| c | c |}
	\hline \textbf{Partição} & \textbf{Processos} \\ \hline
	{\ttfamily 100K} &  \\ \hline
	{\ttfamily 500K} & {\ttfamily P1 e P3} \\ \hline
	{\ttfamily 200K} & \\ \hline
	{\ttfamily 300K} & {\ttfamily } \\ \hline
	{\ttfamily 600K} & {\ttfamily P2} \\ \hline
	\end{tabular}
	\caption{\textsl{First-fit}.}
\end{center}
\end{table}

\begin{table}[!ht]
\begin{center}
	\begin{tabular}{| c | c |}
	\hline \textbf{Partição} & \textbf{Processos} \\ \hline
	{\ttfamily 100K} &  \\ \hline
	{\ttfamily 500K} & {\ttfamily P2} \\ \hline
	{\ttfamily 200K} & {\ttfamily P3} \\ \hline
	{\ttfamily 300K} & {\ttfamily P1} \\ \hline
	{\ttfamily 600K} & {\ttfamily P4} \\ \hline
	\end{tabular}
	\caption{\textsl{Best-fit}.}
\end{center}
\end{table}

\begin{table}[!ht]
\begin{center}
	\begin{tabular}{| c | c |}
	\hline \textbf{Partição} & \textbf{Processos} \\ \hline
	{\ttfamily 100K} &  \\ \hline
	{\ttfamily 500K} & {\ttfamily P2} \\ \hline
	{\ttfamily 200K} & \\ \hline
	{\ttfamily 300K} & {\ttfamily } \\ \hline
	{\ttfamily 600K} & {\ttfamily P1 e P3} \\ \hline
	\end{tabular}
	\caption{\textsl{Worst-fit}.}
\end{center}
\end{table}

\pergunta{5. Por que os tamanhos das páginas são sempre potências de 2?}

Por questões de velocidade.  Para converter o índice de um item em um
\textsl{array} no endereço, é preciso apensar usar operações de 
deslocamento do que multiplicação.  Por isso, um \textsl{byte} tem 8 bits,
que influencia diretamente no modo em como a memória é projetada e no
tamanho das páginas.

\pergunta {6. Explique a diferença entre busca de páginas por demanda e busca 
antecipada. Cite duas estratégias de busca antecipada.}

Na \textbf{paginação por demanda} as páginas dos processos são transferidas para
a memória principal somente	quando são necessárias.

Na \textbf{busca antecipada} o sistema carrega para a memória principal, além
da página necessária, outras páginas que podem ou não ser usadas durante o 
processamento, utilizando assim o princípio da localidade espacial.

\pergunta {7. Por que, em um sistema com paginação, um processo não pode acessar
 áreas de memória que não lhe pertence?}

Para evitar que um bug dentro de um processo afete outro processo, ou até mesmo
o próprio sistema operacional.

\pergunta{8. Em que circunstâncias ocorre uma falha de página? Descreva as ações
tomadas pelo sistema quando uma falha de página ocorre.}

Um \textsl{page fault} ocorre quando um processo tenta acessar um endreço que 
não está na memória principal, no momento em que ocorre, o sistema transfere a 
página da memória secundária para a memória principal.

\pergunta{9. Quando a memória virtual é implementada em um sistema de 
computacão, há
certos custos associados com a técnica e certos benefícios. Liste os custos e os
benefícios. É possível que os custos excedam os benefícios? Se sim, quais
medidas podem ser tomadas para assegurar que isto não aconteca?}


\pergunta{10. Se usarmos um algoritmo de substituição de páginas FIFO com 4 
frames e 8 páginas, quantos page faults vão acontecer na seguinte sequência de 
referências: 0 1 7 2 3 2 7 1 0 3 se os 4 frames estão inicialmente vazios.}

\pergunta{11. Repita o problema anterior usando LRU.}

\pergunta{12. No algoritmo FIFO circular, quais páginas seriam liberadas nas 
situações A e B?}


\end{document}

