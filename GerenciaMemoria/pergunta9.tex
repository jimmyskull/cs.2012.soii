\pergunta{9. Considere que os processos da tabela a seguir estão aguardando para
serem executados e que cada um permanecerá na memória durante o tempo especificado. 
O sistema operacional ocupa uma área de 20Kb no início da memória e gerencia a 
memória utilizando um algoritmo de particionamento dinâmico modificado. 
A memória total disponível no sistema é de 64Kb e é alocada em blocos múltiplos 
de 4Kb.Os processos são alocados de acordo com sua identificação (em ordem crescente) 
e irão aguardar até obter a memória que necessitam. Calcule a perda de
memória por fragmentação interna e externa sempre que um processo é colocado ou 
retirado da memória. O sistema operacional compacta a memória apenas quando existem 
duas ou mais partições livres adjacentes.}

Processo Memória Tempo\\
1 30KB 5\\
2 6KB 10\\
3 36KB 5\\

	
