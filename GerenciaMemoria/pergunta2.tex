\pergunta{2. Considere um sistema computacional com 40KB de memória principal e 
que utilize um sistema operacional de 10KB que implemente alocação contígua de 
memória. Qual a taxa de subutilização da memória principal para um programa que 
ocupe 20KB de memória?}

Assumindo que o sistema operacional ocupará os 10 KB ao início da memória e os
outros 30 KB restantes formem um único bloco contíguo.
Quando o processo que necessita de 20KB de memória for iniciado,  será alocado
o único bloco contíguo de memória, com tamanho 30KB, por ser grande o 
suficiente para acomodá-lo na memória.  Haverá \textbf{10 KB} não utilizados
neste bloco.  Esse espaço perdido é chamado de fragmentação interna 
\cite{ContiguousTechniques}.

