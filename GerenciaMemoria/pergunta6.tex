\pergunta{6. Suponha um sistema computacional com 128Kb de memória principal e que utilize um
sistemaoperacional de 64Kb que implementa alocação particionada estática relocável.
Considere também que osistema foi inicializado com três partições: P1 (8Kb), P2 (24Kb) e P3
(32Kb). Calcule a fragmentaçãointerna da memória principal após a carga de três programas:
PA, PB e PC.}

\pergunta{a) P1 <- PA (6Kb); P2 <- PB (20Kb); P3 <- PC (28Kb)}

\pergunta{b) P1 <- PA (4Kb); P2 <- PB (16Kb); P3 <- PC (26Kb)}

\pergunta{c) P1 <- PA (8Kb); P2 <- PB (24Kb); P3 <- PC (32Kb)}

