\pergunta{6. Suponha um sistema computacional com 128KB de memória principal e que utilize um
sistema operacional de 64KB que implementa alocação particionada estática relocável.
Considere também que o sistema foi inicializado com três partições: P1 (8KB), P2 (24KB) e P3
(32KB). Calcule a fragmentação interna da memória principal após a carga de três programas:
PA, PB e PC.}

\pergunta{a) P1 $\leftarrow$ PA (6KB); P2 $\leftarrow$ PB (20KB); P3 $\leftarrow$ PC (28KB)}

\pergunta{b) P1 $\leftarrow$ PA (4KB); P2 $\leftarrow$ PB (16KB); P3 $\leftarrow$ PC (26KB)}

\pergunta{c) P1 $\leftarrow$ PA (8KB); P2 $\leftarrow$ PB (24KB); P3 $\leftarrow$ PC (32KB)}

