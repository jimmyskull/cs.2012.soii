\documentclass[12pt,a4paper]{article}
\usepackage[utf8]{inputenc}
\usepackage[brazil]{babel}
\usepackage[T1]{fontenc}

\usepackage{times}

\title{Exercícios Sistemas Operacionais II \\ 
	Gerencia de Memória}
\author{Universidade Estadual do Centro-Oeste (UNICENTRO)\\
        Departamento de Ciência da Computação (DECOMP)\\
        Sistemas Operacionais II \\
        Professor Diego Marczal \\
	Paulo R. Urio (RA: 570091403)}

\begin{document}

\maketitle

\noindent \textbf{1. Quais as funções básicas da gerência de memória?}

A função básica do gerenciador de memória é alocar dinamicamente porções
de memória para programas, quando requisitadas, e liberar para reuso 
quando não for mais necessária \cite{IBMMemoryManagement}.  Também deve
gerenciar toda a memória disponível (memória principal e secundária) de 
forma transparente aos programas do \textsl{userspace} 
\cite{http://tldp.org/LDP/tlk/mm/memory.html}.

\bibliographystyle{plain}
\bibliography{main}

\end{document}

