\pergunta{3. Suponha um sistema computacional com 64Kb de memória principal e que utilize um
sistema operacional de 14Kb que implemente alocação contígua de memória. Considere também
um programa de 90Kb, formado por um módulo principal de 20Kb e três módulos
independentes, cada um com 10Kb,20Kb e 30Kb. Como o programa poderia ser executado
utilizando-se apenas a técnica de overlay?}
\\ \\
 90kb -> programa;\\
 10, 20 e 30 kb módulos.\\ 
 64kb  -> computador;\\
-14kb  -> memória utilizada pelo sistema operacional;\\
------\\
 50kb  livres;\\
-20kb  ->modulo principal do programa;\\
------\\
=30kb livres\\
-30kb overlay, já que o overlay possui o mesmo tamanho do maior módulo do programa.\\

Os módulos serão executados na área de overlay.\\
