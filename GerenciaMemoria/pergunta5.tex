\pergunta{5. Qual a diferença entre fragmentação interna e externa da memória 
principal?}

A fragmentação interna ocorre quando os programas não preenchem as partições
onde são carregados, ocorre com técnicas de alocação absoluta, relocável e 
contígua.
A fragmentação externa ocorre em técnicas de alocação dinâmica quando
programas são terminados sem serem completamente liberados da memória, deixando
cada vez espaços menores na memória, assim novos programas não podem ser executados.

