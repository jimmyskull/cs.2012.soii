\pergunta{5. Qual a diferença entre fragmentação interna e externa da memória 
principal?}

A fragmentação interna ocorre quando os programas não preenchem as partições
onde são carregados, ocorre nos sistemas de alocação absoluta e nos sistemas
de alocação relocável.
A fragmentação externa ocorre nos sistemas de alocação dinâmica ocorre quando
programas são terminados mas não são completamente liberados da memória, deixando
cada vez espaços menores na memória, assim novos programas não podem ser executados.\\
